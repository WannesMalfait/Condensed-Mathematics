\documentclass{article}

\input{definitions/preamble}
\input{definitions/macros}
\input{definitions/letterfonts}

\DeclareMathOperator{\Open}{\mathbf{Open}}
\DeclareMathOperator{\opp}{opp}
\DeclareMathOperator{\Cont}{Cont}
\DeclareMathOperator{\Cov}{Cov}
\DeclareMathOperator{\colim}{colim}
\DeclareMathOperator{\id}{id}
\DeclareMathOperator{\Cond}{Cond}
\DeclareMathOperator{\CHaus}{\mathbf{CHaus}}
\renewcommand{\top}{\textrm{top}}

\title{A Condensed Introduction to Condensed Mathematics}
\author{Wannes Malfait}
\date{}

\begin{document}

\maketitle


\section{Introduction}

The goal of this paper is to provide
an idea of what condensed mathematics is about,
while not assuming too\footnote{
    Throughout this paper it will be assumed that the reader
    is somewhat familiar with basic notions from
    category theory, and in particular homological
    algebra.
} much mathematical background
from the reader.
The material is based on the lecture
notes by Peter Scholze and the master class in
condensed mathematics by
Peter Scholze and Dustin
Clausen (\cite{Sch2019LecturesCM,Sch2020MasterClass}).
Let us now try to motivate why one might
consider studying condensed sets. Additionally,
a potential train of thought leading to the
definition of a condensed set is proposed, which
hopes to give the reader some intuition where
this seemingly strange definition comes from.

\section{Why condensed sets?}

The modern definition of a topological space
only came about in the 1920s. It has proven
to be a very interesting and rich field of study,
with many applications across mathematics. Even so,
it seems that there is something fundamentally wrong
with the way we combine algebraic structures with
topology. Consider the inclusion map
\begin{equation*}
    \bbQ \injto \bbR,
\end{equation*}
for example. As a map between abelian groups,
this map is injective, but not surjective. However,
when we instead consider this as a map between
Hausdorff topological abelian groups, where $\bbQ$ and $\bbR$
have been equipped with their usual metric topology,
then this map also becomes an epimorphism. Indeed,
if two continuous maps $f,g\colon \bbR \to X$ agree
on the dense subspace $\bbQ$, then they must be equal,
if $X$ is Hausdorff. One sees that it somehow becomes
more difficult to tell apart $\bbQ$ from $\bbR$
in the topological setting.

A potential way to fix
this would be to consider only a particularly nice
class of topological spaces. For example, one could
try to restrict to just $\CHaus$, the category
of compact Hausdorff spaces. There, every continuous
map is also closed, and hence a continuous bijection
is automatically a homeomorphism. In fact, this
category is an abelian category. Indeed, Pontryagin
duality gives that discrete abelian groups
are dual to compact Hausdorff groups (by considering
the group of continuous group characters). So
$\Ab^{\opp} \simeq \CHaus$, and $\CHaus$ is
abelian as the opposite category of an abelian category.
This means that it makes sense to start talking
about long exact sequences and homology in this
setting.
The problem with
this approach is that there are many interesting
and non-pathological topological
groups that are not compact. For example, the real
numbers with the metric topology is not compact.
Hence, we will need some way of including a
broader class of topological spaces.

At this point it is a good idea to study the literature.
The field of algebraic geometry turns out to be particularly
fruitful. In \cite{Steenrod1967ConvenientCT} it is suggested
to look at the category of compactly generated
spaces. This category satisfies some nice properties
relating unions products and quotients, such as $Z^{Y\times X} = (Z^Y)^X$.
It is also not too strict: it contains
many families of topological spaces, such as locally compact
spaces, CW-complexes and first-countable spaces. The product in this
category turns out to be the natural product for CW-complexes
(see the appendix of \cite{Hat2002AlgebraicT}).
What does it mean for a topological space to be compactly generated?
\begin{definition}
    A topological space $X$ is said to be \emph{compactly generated}
    if the topology on $X$ coincides with the final topology generated
    by all the continuous maps $K \to X$ with $K$ compact Hausdorff.
    In other words, a map $f\colon X \to Y$ is continuous, if
    $K \to X \to Y$ is continuous for every continuous map $K \to X$.
\end{definition}
So, if we understand the continuous maps $K \to X$, we understand
the space $X$. We now take a small leap in logic, motivated by algebraic
geometry, and conclude that $X$ is determined by the functor
\begin{equation*}
    \underline{X} \defeq \Cont(-, X) \colon \CHaus^{\opp} \to \Set,
\end{equation*}
which maps any compact Hausdorff space $K$ to the set of
continuous functions $K \to X$. This claim is justified in
\cref{prop:compactly_generated_adjunction}. We can of course
consider the same type of functor for any topological space
$X$. The functor $\underline{X}$ (with some minor modifications)
will be the condensed set
associated to $X$. For this to lead to a good theory, which
would replace topological spaces, we want two things:
\begin{enumerate}
    \item The definition is strong enough. We are able
          to express the topological properties we want in this
          new setting.
    \item The definition is not too strong. We do not
          end up with a theory that is just the same as topological spaces.
          Constructions and invariants that we care about should be
          valid in this new setting.
\end{enumerate}
The goal of the following sections is to provide evidence
that condensed mathematics does indeed behave as wanted.

\section{What are condensed sets?}

Recall from the previous section that we moved from
topological spaces $X$, to contravariant functors $\underline{X}$.
We now want to view these as objects of some category.
As morphisms,
i.e. maps between condensed sets, we'll take natural transformations.
This poses a few problems.

Firstly,
we need to be careful to not end up with set-theoretical
issues. Indeed, the category $\CHaus$ has a proper class
of objects, and hence the natural transformations between
two functors $\underline{X}$ and $\underline{Y}$ will fail
to be a set in general. To solve this, we fix some
uncountable strong limit cardinal $\kappa$. This is
a cardinal $|\bbN| < \kappa$ such that $2^\lambda < \kappa$ for any
cardinal $\lambda < \kappa$. This means that
if $X$ and $Y$ are sets with cardinality smaller than $\kappa$,
then so will their product, union, and respective powersets.
We will then restrict to just those compact Hausdorff spaces
which are $\kappa$-small, i.e. with cardinality less than $\kappa$.
In practice this is not a big restriction. Most sets one encounters
``in the wild''
have cardinality less than $\kappa$. For example, the set of
real numbers, or even the set of functions $\bbR \to \bbR$ all have
cardinality less than $\kappa$. Indeed, $|\bbR| = 2^{|\bbN|} < \kappa$
and
\begin{equation*}
    |\bbR^{\bbR}| \leq 2^{|\bbR\times \bbR|}
    = 2^{|\bbR|} < \kappa.
\end{equation*}
With this, we get a set of natural transformations between
any two functors.

The second problem is that the category doesn't have all
the nice properties that we want yet. For example,
if we take two topological spaces $X$ and $Y$, and
consider the associated condensed sets $\underline{X}$
and $\underline{Y}$, then what would be the coproduct
$\underline{X} \coprod \underline{Y}$? This would need
to be some functor $\underline{Z}$. TODO: work out this
example or come up with a better one.
So, we will need to consider more objects, than just
the \emph{representable} functors $\underline{X}$.
From the example it becomes apparent that we need
to somehow add colimits of the representable functors
to our category. We can once again reap our solution from
the field of algebraic topology: sheaves\footnote{
    It can be shown that a sheaf is a colimit of (sheafifications of) representable
    sheaves (see
    \cite[\href{https://stacks.math.columbia.edu/tag/0GLW}{Lemma 0GLW}]{stacks-project}).
}\footnote{
    The pun with agricultural sheaves and fields \emph{is} intended.
}. Sheaves are a very general object, and appear in various
contexts. A condensed set is a special type of sheaf.
We omit the definition of what a sheaf is here, and instead
describe what it means in this context. For the interested
readers, \cite[Section 1.2]{Dag2021FoundationsCM} gives
a detailed description of condensed sets in terms of sheaves
on a site.
\begin{definition}
    A $\kappa$-\emph{condensed set} is a functor
    \begin{equation*}
        T \colon \CHaus_{\kappa}^{\opp} \to \Set,
    \end{equation*}
    such that $T(\emptyset) = \{*\}$, and
    \begin{enumerate}
        \item For any two compact Hausdorff spaces $K_1$ and $K_2$ with cardinality
              smaller than $\kappa$, the natural map
              \begin{equation*}
                  T(K_1 \sqcup K_2) \to T(K_1) \times T(K_2),
              \end{equation*}
              is a bijection.
        \item For any surjection $K' \surjto K$ of compact Hausdorff spaces
              with cardinality smaller than $\kappa$, let $p_1$ and $p_2$
              be the two projections $K'\times_K K' \to K'$. Then the map
              \begin{equation*}
                  T(K) \to \{x \in T(K') \mid Tp_1(x) = Tp_2(x)\},
              \end{equation*}
              is also a bijection.
    \end{enumerate}
\end{definition}

Let us now try to unpack this definition.
For the first condition, note that there are two inclusion maps
$K_1 \to K_1 \sqcup K_2$ and $K_2 \to K_1 \sqcup K_2$, which give
maps $T(K_1 \sqcup K_2) \to T(K_1)$ and $T(K_1 \sqcup K_2) \to T(K_2)$.
By the universal property of the product we obtain a map
$T(K_1 \sqcup K_2) \to T(K_1) \times T(K_2)$.

In the second property, let $f\colon K' \surjto K$ be the surjection.
We then have the fiber product $K' \times_K K' = \{(x,y)\in K' \times K' \mid f(x) = f(y)\}$,
together with the projection maps $p_1, p_2$ onto the two components.
Since $f\circ p_1 = f \circ p_2$, we have $Tp_1 \circ Tf = Tp_2 \circ Tf$
which shows that the map in the definition is well-defined.

Let us verify that the functors $\underline{X}$ restrict
to $\kappa$-condensed sets. Since there is a unique function
$\emptyset \to X$, we see that $T(\emptyset) = \{*\}$ as this function
is also continuous. For the other two properties:
\begin{enumerate}
    \item By the universal property of the coproduct, giving
          a continuous map $K_1 \sqcup K_2 \to X$ is the same as giving
          a pair of
          continuous maps $K_1 \to X$ and $K_2 \to X$.
          So $\Cont(K_1 \sqcup K_2, X) \to \Cont(K_1, X) \times \Cont(K_2, X)$
          is a bijection.
    \item Let $e\colon K' \surjto K$ be a surjection of $\kappa$-small
          compact Hausdorff spaces. This gives a map
          \begin{equation*}
              Te \colon \Cont(K, X) \to \Cont(K', X) \colon g \mapsto g\circ e.
          \end{equation*}
          Since $e$ is a surjection, the map $Te$ is an injection as
          $g_1\circ e = g_2 \circ e$ implies $g_1 = g_2$. We now need
          to show that the image of $Te$ is precisely
          \begin{equation*}
              S = \{h \in \Cont(K' , X) \mid h\circ p_1 = h\circ p_2\}.
          \end{equation*}
          We already saw that $Te \subseteq S$. On the other hand,
          if $h \in S$, then $e(x) = e(y) \implies h(x) = h(y)$ for
          any $x,y\in K'$. This implies that there is a well-defined
          map of sets $s\colon K \to X$ with $s(k) = h(e\inv(\{k\}))$.
          By construction, $h = s\circ e$. Since $e$ is a surjection
          of compact Hausdorff spaces, it is closed. So $s$ is continuous,
          as for any closed $W \subseteq X$, $s\inv(W) = e(h\inv(W))$
          is closed. This shows that $h\in Te$.
\end{enumerate}
TODO: explain intuition about these properties.

This gives some confidence that the definition of a $\kappa$-condensed set
is a good one. We have now seen that we can go from a topological space
to a condensed set by sending $X$ to $\Cont(- ,X)$. Is there also a way
to retrieve a topological space from a condensed set?
From a condensed set $T$, we can obtain the ``underlying set'', by evaluating
the functor in the singleton: $T(\{*\})$. Of course, not all condensed sets
are of the form $\Cont(-, X)$ for some topological space $X$, and there are
non-trivial condensed sets for which $T(\{*\}) = \{*\}$ as we will
see later on. The question still remains how we can equip the underlying
set with a topology. We arrived at the definition of a
condensed set by looking at compactly
generated spaces. This also gives us the intuition for how to
define a topology on the underlying set.

Let $T$ be a $\kappa$-condensed
set, and $K$ a $\kappa$-small compact Hausdorff space. First, note that every
element $x \in T(K)$ induces a map (of sets) $f_x \colon K \to T(\{*\})$. Indeed,
for every $k\in K$ we have an inclusion map $i_K \colon \{k\} \to K$
which gives a map $Ti_k \colon T(K) \to T(\{k\}) \cong T(\{*\})$. Then
$f_x(k)$ is simply defined as $Ti_k(x)$. To make this clearer, we should
look at the special case $T = \underline{X}$ for some topological space
$X$. Let $h\in \underline{X}(K) = \Cont(K , S)$, then $f_h = h$.
After all, $f_h(k) = \underline{X}i_k(h) = h\circ i_k = h(k)$.
So, we can think of the elements of $T(K)$ as functions $K \to T(\{*\})$.
\begin{definition}
    Let $T$ be a $\kappa$-condensed set. We equip the underlying set $T(\{*\})$
    with the final topology induced by all the maps $f_x \colon K \to T(\{*\})$
    for every $\kappa$-small compact Hausdorff space $K$ and $x\in K$.
    This topological space is denoted as $T(\{*\})_\top$.
\end{definition}

Let us call a topological space $\kappa$-compactly generated, if
it is compactly generated by just the $\kappa$-small compact Hausdorff spaces.
From the way we constructed the topology on the underlying set
we find that if $X$ is $\kappa$-compactly generated,
then $\underline{X}(\{*\})_\top \cong X$. We can say more:
\begin{prop}
    \label{prop:compactly_generated_adjunction}

\end{prop}

TODO: explain how it works with extremally disconnected.
\section{Condensed abelian groups}
\bibliographystyle{alpha}
\bibliography{references.bib}
\end{document}
