\documentclass{article}

\input{definitions/preamble}
\input{definitions/macros}
\input{definitions/letterfonts}

\DeclareMathOperator{\Open}{\mathbf{Open}}
\DeclareMathOperator{\opp}{opp}
\DeclareMathOperator{\Cont}{Cont}
\DeclareMathOperator{\Cov}{Cov}
\DeclareMathOperator{\colim}{colim}
\DeclareMathOperator{\id}{id}
\DeclareMathOperator{\Cond}{Cond}
\DeclareMathOperator{\CHaus}{\mathbf{CHaus}}

\title{A Condensed Introduction to Condensed Mathematics}
\author{Wannes Malfait}
\date{}

\begin{document}

\maketitle


\section{Introduction}

The goal of this paper is to provide
an idea of what condensed mathematics is about,
while not assuming too\footnote{
    Throughout this paper it will be assumed that the reader
    is somewhat familiar with basic notions from
    category theory, and in particular homological
    algebra.
} much mathematical background
from the reader.
The material is based on the lecture
notes by Peter Scholze and the master class in
condensed mathematics by
Peter Scholze and Dustin
Clausen (\cite{Sch2019LecturesCM,Sch2020MasterClass}).
Let us now try to motivate why one might
consider studying condensed sets. Additionally,
a potential train of thought leading to the
definition of a condensed set is proposed, which
hopes to give the reader some intuition where
this seemingly strange definition comes from.

\section{Why condensed sets?}

The modern definition of a topological space
only came about in the 1920s. It has proven
to be a very interesting and rich field of study,
with many applications across mathematics. Even so,
it seems that there is something fundamentally wrong
with the way we combine algebraic structures with
topology. Consider the inclusion map
\begin{equation*}
    \bbQ \injto \bbR,
\end{equation*}
for example. As a map between abelian groups,
this map is injective, but not surjective. However,
when we instead consider this as a map between
Hausdorff topological abelian groups, where $\bbQ$ and $\bbR$
have been equipped with their usual metric topology,
then this map also becomes an epimorphism. Indeed,
if two continuous maps $f,g\colon \bbR \to X$ agree
on the dense subspace $\bbQ$, then they must be equal,
if $X$ is Hausdorff. One sees that it somehow becomes
more difficult to tell apart $\bbQ$ from $\bbR$
in the topological setting.

A potential way to fix
this would be to consider only a particularly nice
class of topological spaces. For example, one could
try to restrict to just $\CHaus$, the category
of compact Hausdorff spaces. There, every continuous
map is also closed, and hence a continuous bijection
is automatically a homeomorphism. In fact, this
category is an abelian category. Indeed, Pontryagin
duality gives that discrete abelian groups
are dual to compact Hausdorff groups (by considering
the group of continuous group characters). So
$\Ab^{\opp} \simeq \CHaus$, and $\CHaus$ is
abelian as the opposite category of an abelian category.
This means that it makes sense to start talking
about long exact sequences and homology in this
setting.
The problem with
this approach is that there are many interesting
and non-pathological topological
groups that are not compact. For example, the real
numbers with the metric topology is not compact.
Hence, we will need some way of including a
broader class of topological spaces.

At this point it is a good idea to study the literature.
The field of algebraic geometry turns out to be particularly
fruitful. In \cite{Steenrod1967ConvenientCT} it is suggested
to look at the category of compactly generated
spaces. This category satisfies some nice properties
relating unions products and quotients, such as $Z^{Y\times X} = (Z^Y)^X$.
It is also not too strict: it contains
many families of topological spaces, such as locally compact
spaces, CW-complexes and first-countable spaces. The product in this
category turns out to be the natural product for CW-complexes
(see the appendix of \cite{Hat2002AlgebraicT}).
What does it mean for a topological space to be compactly generated?
\begin{definition}
    A topological space $X$ is said to be \emph{compactly generated}
    if the topology on $X$ coincides with the final topology generated
    by all the continuous maps $K \to X$ with $K$ compact Hausdorff.
    In other words, a map $f\colon X \to Y$ is continuous, if
    $K \to X \to Y$ is continuous for every continuous map $K \to X$.
\end{definition}
So, if we understand the continuous maps $K \to X$, we understand
the space $X$. We now take a small leap in logic, motivated by algebraic
geometry, and conclude that $X$ is determined by the functor
\begin{equation*}
    \underline{X} \defeq \Cont(-, X) \colon \CHaus^{\opp} \to \Set,
\end{equation*}
which maps any compact Hausdorff space $K$ to the set of
continuous functions $K \to X$. This claim is justified in
\cref{prop:compactly_generated_adjunction}. We can of course
consider the same type of functor for any topological space
$X$. The functor $\underline{X}$ (with some minor modifications)
will be the condensed set
associated to $X$. For this to lead to a good theory, which
would replace topological spaces, we want two things:
\begin{enumerate}
    \item The definition is strong enough. We are able
          to express the topological properties we want in this
          new setting.
    \item The definition is not too strong. We do not
          end up with a theory that is just the same as topological spaces.
          Constructions and invariants that we care about should be
          valid in this new setting.
\end{enumerate}
The goal of the following sections is to provide evidence
that condensed mathematics does indeed behave as wanted.

\section{What are condensed sets?}

\begin{prop}
    \label{prop:compactly_generated_adjunction}
\end{prop}

\section{Condensed abelian groups}
\bibliographystyle{alpha}
\bibliography{references.bib}
\end{document}
